\documentclass[12pt]{article}
\usepackage[english]{babel}
\usepackage{natbib}
\usepackage{url}
\usepackage[utf8x]{inputenc}

\usepackage{parskip}
\usepackage{fancyhdr}
\usepackage{vmargin}
\setmarginsrb{3 cm}{2.5 cm}{3 cm}{2.5 cm}{1 cm}{1.0 cm}{1 cm}{1.5 cm}

\usepackage[linktocpage=all]{hyperref}
\usepackage{color}   %May be necessary if you want to color links
\usepackage{hyperref}
\hypersetup{
    colorlinks=true, %set true if you want colored links
    linktoc=all,     %set to all if you want both sections and subsections linked
    linkcolor=black,  %choose some color if you want links to stand out
}

\usepackage{float}
\usepackage{caption}
\usepackage{subcaption}
\usepackage{parcolumns}

\usepackage{amsmath}
\usepackage{amssymb}
\usepackage{amsthm}

\usepackage{graphicx}
\graphicspath{{images/}}

\usepackage{booktabs}
\usepackage{bm}

\usepackage{pdfpages}
\usepackage{tocloft}

\title{Group Assignment}								% Title
\author{
    % TODO: PLEASE INSERT YOUR UNI KEYS HERE FOLLOWED BY '\\'
	450411920 \\
}								% Author
\date{Semester 1, 2019}											% Date

\makeatletter
\let\thetitle\@title
\let\theauthor\@author
\let\thedate\@date
\makeatother

\pagestyle{fancy}
\fancyhf{}
%\rhead{\theauthor}
\lhead{Big Data Tuning Assignment}
\cfoot{\thepage}



\begin{document}

%%%%%%%%%%%%%%%%%%%%%%%%%%%%%%%%%%%%%%%%%%%%%%%%%%%%%%%%%%%%%%%%%%%%%%%%%%%%%%%%%%%%%%%%%

\begin{titlepage}
	\centering
    \vspace*{0.5 cm}
    \includegraphics[scale = 0.75]{USYD_LOGO_New.jpg}\\[1.0 cm]	% University Logo
    \textsc{\LARGE University of Sydney}\\[2.0 cm]	% University Name
	\textsc{\Large DATA3404}\\[0.5 cm]				% Course Code
	\textsc{\large Data Science Platforms}\\[0.5 cm]				% Course Name
	\rule{\linewidth}{0.2 mm} \\[0.4 cm]
	{ \huge \bfseries \thetitle}\\
	\rule{\linewidth}{0.2 mm} \\[1.5 cm]
	
	
		
	\emph{Team members:}\\
	\theauthor
		
	\begin{minipage}{0.45\textwidth}
			
	\end{minipage}\\[2 cm]
	
	{\large \thedate}\\[2 cm]
 
	\vfill
	
\end{titlepage}

%%%%%%%%%%%%%%%%%%%%%%%%%%%%%%%%%%%%%%%%%%%%%%%%%%%%%%%%%%%%%%%%%%%%%%%%%%%%%%%%%%%%%%%%%

\tableofcontents
\pagebreak

%%%%%%%%%%%%%%%%%%%%%%%%%%%%%%%%%%%%%%%%%%%%%%%%%%%%%%%%%%%%%%%%%%%%%%%%%%%%%%%%%%%%%%%%%

\phantomsection
\addcontentsline{toc}{section}{Job Design Documentation}
\section*{Job Design Documentation}
\addcontentsline{toc}{subsection}{Task 1}
\subsection*{Task 1: Top 3 Cessna Models}
We imported the csv files containing the relevant fields. We then joined the ontimeperformance\_flights with ontimeperformance\_aircrafts and projected the relevant columns. We then applied a filter function to only return aircrafts with the model equaled to "CESSNA". Finally, we applied a flatmap to count all the instances of the Cessna model, rank them in descending order, and return the top 3 Cessna model aircrafts.
\addcontentsline{toc}{subsection}{Task 2}
\subsection*{Task 2}
Blah blah.
\addcontentsline{toc}{subsection}{Task 3}
\subsection*{Task 3}
Join join join.
\newpage{}

\phantomsection
\addcontentsline{toc}{section}{Tuning Decisions and Evaluation}
\section*{Tuning Decisions and Evaluation}
\addcontentsline{toc}{subsection}{Task 1}
\subsection*{Task 1: Top-3 Cessna Models}
Blah blah blah
\addcontentsline{toc}{subsection}{Task 2}
\subsection*{Task 2: Average Departure Delay}
We use a broadcash hash join if a file is significantly smaller than another. The rationale is that a hash join will be executed whereby the data will be split into buckets, that are later merged. This achieves a speed up in run time in comparison to traditional loop joins.
\addcontentsline{toc}{subsection}{Task 3}
\subsection*{Task 3: Most Popular Aircraft Types}
Join join join.
\newpage{}


\phantomsection
\addcontentsline{toc}{section}{Performance Evaluation}
\section*{Performance Evaluation}
\addcontentsline{toc}{subsection}{Task 1: Top-3 Cessna Models}
\subsection*{Task 1}
blah blah blah
\addcontentsline{toc}{subsection}{Task 2: Average Departure Delay}
\subsection*{Task 2}
Blah blah blah 
\addcontentsline{toc}{subsection}{Task 3}
\subsection*{Task 3: Most Popular Aircraft Types}
Join join join.
\newpage{}

\end{document}